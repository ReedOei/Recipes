\documentclass[10pt]{article}

\usepackage{amsmath}
\usepackage{hyperref}
\usepackage{tikz-cd}
\usepackage{amssymb}
\usepackage{amsthm}
\usepackage{bm}
\usepackage{listings}
\usepackage{bbm}
\usepackage{multicol}
\usepackage{mathtools}
\usepackage{mathpartir}
\usepackage{float}
\usepackage{enumerate}
\usepackage[margin=1.25in]{geometry}
\usepackage[T1]{fontenc}
\usepackage{kpfonts}

\input{../Reed/School/LaTeX/macros.tex}
\input{../Reed/School/LaTeX/lemmas.tex}

\begin{document}

\subsection*{Sauces}

\subsubsection*{A simple tomato paste sauce}

Take a chopped onion and put in a small pot with some olive oil, and cook until translucent.
Add garlic and cook another minute.
Put in the tomatoes, which should be cut into some pieces---it will cook down, so it doesn't matter much how small.
Smash the tomatoes, and add salt, pepper, oregano, basil, parsley, sage, and some red pepper flakes.
Let simmer until done.

\subsubsection*{A pasta sauce with meat}

Prepare the base of the sauce as above, or use any other sauce you please.

To prepare the meat, cook with an onion and chop the meat into small bits as it cooks.
Once it is browned, drain as much of the grease as you wish, and pour in the sauce base.
Simmer for a quarter hour (or a bit more) until combined.

Ground beef works well as the meat for this recipe, as does a combination of ground beef and sausage.

\subsubsection*{To stuff peppers}

This recipes is intended for poblano peppers, but it is likely any sizeable and sturdy pepper would work.

Start by cooking the wild rice and broiling the poblano pepper on both sides.
When poblanos are done, peel the outer layer of the skin.

When both of these are nearly done, start on the filling.
Cook onions with oil until nearly translucent, then add garlic and cook a little more.
Then add tomatoes and cook until softened and a bit reduced.
Add corn, black beans, and wild rice.
Cook a little, then add spices: cayenne pepper, cumin, chili powder, paprika, salt, and pepper.

Bring the over to a good roasting temperature, say 400 degree Fahrenheit.
Sprinkle some lime over the poblanos, and add cheese---both optional.
Put stuffed peppers into the oven for 10-15 minutes.
Once done, take them out and serve them forth.

\subsection*{Tofu}

\subsubsection*{To fry tofu}

Drain the tofu and place on a (paper) towel on a plate, with another towel and weights (more plates works well) on top.
Wait some time for the liquid to come out---you may wish to repeat this process.

Cut the tofu into chunks, salt them, and fry in enough hot oil so that the sides are covered.
Be patient to let them get to a good color before you flip them to fry on both sides.

\subsection*{Soups, Stews, and the Like}

\subsubsection*{The basic method for stewing}

This works well for several kinds of meat, including beef and pork (the author's preferred meats, rather than, say, chicken).

Take a small bowl of flour and add salt, then flour meat, which has been trimmed as you please and cut into bite-sized pieces.
Brown the meat in the same pot you intend to cook the stew in, with a generous helping of butter in the pot.
You may use vegetable oil instead, but it will be worse.
Remove the meat once browned on two sides and set aside.

Add an onion (and celery, if desired/it is to be had) and cook until translucent and soft; don't let it brown.
Add a generous amount of garlic and cook for a minute or two.
Once this is done, add whatever herbs and spices you intend to use---for an herby flavor, oregano, thyme, and rosemary work well.
Almost every stew benefits from a couple of bay leaves.
Let this cook a minute, then add the wine.
The type of wine depends on the meat: pair as you usually would.

Once the wine has reduced a bit, add the meat and the broth, and bring to a boil.
Now is the time to skim fat from the top of the stew.
Simmer for about an hour.
Then add whatever vegetables you please (potatoes and carrots are always a welcome addition), and simmer for another half hour, or until the vegetables are done to your liking.

\subsubsection*{To make chili the Texas way}

This mostly follows the basic method for stewing as described earlier.

Trim some of the fat off of a 2 lbs. chuck roast and cut into slightly-large-than-bite-sized pieces.
Salt and pepper the meat, then brown in the pot with a generous amount of butter.
This may take multiple batches, and simply set aside the meat when it is browned on two sides.
Reserve some for eating while you cook!

Once the meat is browned, add very roughly cut onions and green bell peppers (about one large onion and two medium bell peppers).
Cook until the onions are soft and translucent, then add chopped jalape\~{n}os (about three for a pleasant but noticeable spiciness) with seeds removed.
Cook for a couple more minutes, then add garlic, and couple for another minute or two.
Then add the chili powder, made from about 8 large chilis (a mixture of pasilla, chipotle, and ancho work well).
Also add a couple tablespoons of paprika and cumin, and about half a tablespoon of coriander powder.
Mix and cook for a minute to let the spices cook.

Then add the beer and beef broth, and the meat that was set aside earlier, along with all of its juices.
Bring to a boil, skimming the fat off the top.
Then simmer for about 3 hours.

Serve with chips, sour cream, and cheese.

\subsubsection*{A lentil dish with kale}

Boil a good handful (about 3/2 cups) of the washed yellow split peas with a teaspoon of tumeric and a good bit of salt for an hour and a quarter.
As it comes to a boil, skim the water of the starch.
Ensure you have enough water so that the peas will absorb it all, about ten times as much water as lentils---it should cover the dry lentils.

After an hour of boiling, begin to make the kale mixture that will go into the lentils.
Coat a pan with enough olive oil and fry the onion for a little bit.
Then add a lot of garlic and spices (a teaspoon of cumin, the same of charnuska, two of garam masala, and a red pepper flakes to your liking).
If you wish, add a good spoonful of tomato paste.
Fry all this together for a couple of minutes, then add the kale and a small amount of water, just enough to fill the pan.

Once the water has completely evaporated from the kale mixture and the lentils are ready, combine the two and cook another few minutes to combine.
Serve with rice.

\subsubsection*{Risotto}

Melt a goodly chunk of butter in the pot and cook the chopped onion in it until translucent.
Add the arborio rice and cook, taking care it does not brown, until the rice is the same.
Cover well with white wine (pinot grigio is my preference) and stir until the wine has been absorbed---that is, it leaves only slight residue of liquid when cleared from a spot in the pot.

Then, cover well with broth (chicken or vegetable, as you prefer) and stir continuously until it has been absorbed in the same way.
Repeat this with the broth two or three times, stirring all the while, until the risotto is creamy; drink the rest of the wine while you stir.

Add finely grated cheese and cook for a few more minutes, then add a little nutmeg and black pepper.
Stir this in, and finally add any other ingredients you please.

Mushrooms and asparagus work well in this regard: to prepare them, put a chunk of butter and a little olive oil in a pan, cover with the mushrooms, then with the asparagus.
This will let the juices out of the mushrooms which will cook both them and the asparagus.
Cook until the mushrooms have browned, and add to the risotto.

\subsection*{Food for Breakfast}

You can eat any food any time (you can send anyone who disagrees to me), but these are the traditional ``breakfast foods.''

\subsubsection*{To make 5 Pancakes}

To 1/2 cup flour put 2/5 tsp of salt, 1 tsp of baking powder, and some sugar.
Mix.
Add an egg, 1/4 cup of milk, a tablespoon of butter and vanilla (if you please).
Mix.
Spoon into equal portions and cook over medium heat in a lightly oiled pan.

\subsubsection*{How to make buttered eggs}

This recipe is 90\% technique, and 10\% ingredients.
It works best for 1-3 eggs, and if you wish to make more, it is best done in multiple batches.
But your eggs are your eggs and you may do as you please.
The goal is to make eggs that have a just-barely-firm outside and a slightly runny and creamy inside.

Take a tablespoon of butter per two eggs and melt in a pan which only so large as to have the eggs form a thin layer over the entire pan.
Spread butter evenly across pan as it melts, and when it is melted, remove from the heat, and add the eggs.
Mix the eggs with a spatula after adding a pinch of salt (you may do this in a bowl first, but this takes fewer dishes).
Take care not to mix too well---there should be visibly separate bits of white and yolk still.
Return the pan to the heat, and let sit briefly until the eggs start to solidify on the bottom.
Then, using the spatula, scrape the eggs from the bottom of the pan frequently, never letting them sit for too long.
Fold the eggs in on themselves so they begin to come together into a coherent lump.
Once they have just done so, turn off the heat, and continue folding until the bottom of the eggs has just become firm.
Then flip the eggs over briefly, and serve immediately.

This recipe is only exceptional so long as the eggs are warm.

\subsubsection*{Frying potatoes}

Take about one medium potato per person.
These should be precooked, either baked in the oven as a baked potato (leftover baked potatoes work well for this), or in the microwave for about 3 minutes per side (add 15-30 seconds per potato, including the first).
If done in the microwave, poke with a fork before microwaving.

Then cut the potatoes into small cubes, and put into a pan with a generous layer of a mixture of melted butter and olive oil; the pan should be large enough to leave some space between the potatoes.
Season with salt and pepper.
Then cook, taking care not to stir too often so that a good crust can form on the potatoes.
If you wish, you may add onions (about half an onion per potato), garlic (lots), or both several minutes after the potatoes have begun frying, as they will burn otherwise.
Near the end, you may also add paprika.

Once potatoes are well-browned on multiple sides, remove from heat and serve.

\subsubsection*{Frying Sweet Potatoes}

These may be done as regular potatoes, though they tend to be somewhat tougher and require more microwaving.
Take extra time to brown them to draw out the flavors.

\subsubsection*{Applesauce}

You will need a vessel with a lid that can be microwaved, and as many apples, of whatever variety you please, as will fit, unchopped, into the vessel.
This recipe assumes your vessel fits roughly ten medium apples.
Peel and roughly chop the apples.
Add a small amount of water to the vessel.
Then add a layer of apples, cover with a generous amount of sugar and cinnamon, and a small amount of nutmeg.
Continue in this way until the vessel is full, and add the juice of one lemon.

Then, place vessel in microwave and cook in four to six minute increments, stirring between each.
Once the apples are softened to your liking, mash them, and serve it forth.

\subsubsection*{French Toast}

To make the batter, beat well two eggs with a bit of sugar and half a cup of flour and twice that of flour.
Take the bread, which need not be of high quality, cut into large pieces (as you please) and soak it in the batter.
When it has briefly soaked, taking care that it does not fall apart, place it directly into the pan of hot vegetable oil or shortening.
Shortening is to be preferred, but it makes little difference.
Fry both sides of the bread, and when they are both of good color, remove from the pan.

Serve hot with powdered sugar and maple syrup.

Excess batter is good fried on its own in small pieces until crunchy.

\subsection*{Baked Goods}

\subsubsection*{The Best Pie Crust}

Mix well two cups of flour with a teaspoon of salt.
Add to the flour half a cup of vegetable oil and half that amount of milk.
Mix and form a ball.
The dough should not be dry, and she feel lightly oiled.

This amount is sufficient for the bottom or top of the usual size of pie, leaving enough for a small cookie.
This cookie is good with cinnamon and sugar baked on its own alongside your pie.

\subsubsection*{Christmas Sugar Cookies}

Cream together a cup of butter and 5/8 cups of sugar.
Slowly add two and half cups of flour, then refrigerate the dough for some time.

Roll it out thinly, and decorate with colored sugar crystals.
Bake at 325 for about a quarter hour.

You will wish you had made two or three times this amount.

\subsection*{Snacks}

\subsubsection*{Popcorn}

Make a layer of oil, mostly popcorn oil with a little olive oil, in a pot which will fully cover a popcorn kernel.
Put in a popcorn kernel and a goodly amount of salt, and make the pot quite hot.
When the kernel pops, add kernels to mostly cover the bottom of the pot (as the height of your pot permits).
Cover and wait for the kernels to begin to pop.
As they do, shake the pot to coat all the popped kernels evenly.

\subsubsection*{Sweet Fried Chickpeas}

Take prepared chickpeas and mix well with paprika, brown sugar, a little cayenne pepper, and salt.
Put them into a pot of oil so the oil will cover them, and let fry for about a quarter hour, until they are just a little browned.

\subsection*{Other}

\subsubsection*{Egg Salad}

Boil eggs.
Chop them up and mix in a bowl with dried mustard, celery and onion chopped very fine, salt, pepper, and mayonnaise.
Serve on toasted bread.

\end{document}

