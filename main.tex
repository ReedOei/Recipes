\documentclass[10pt]{article}

\usepackage{amsmath}
\usepackage{hyperref}
\usepackage{tikz-cd}
\usepackage{amssymb}
\usepackage{amsthm}
\usepackage{bm}
\usepackage{listings}
\usepackage{bbm}
\usepackage{multicol}
\usepackage{mathtools}
\usepackage{mathpartir}
\usepackage{float}
\usepackage{enumerate}
\usepackage[margin=1.25in]{geometry}
\usepackage[T1]{fontenc}
\usepackage{kpfonts}

\input{../Reed/School/LaTeX/macros.tex}
\input{../Reed/School/LaTeX/lemmas.tex}

\begin{document}

\subsubsection*{To make 5 Pancakes}

To 1/2 cup flour put 2/5 tsp of salt, 1 tsp of baking powder, and some sugar.
Mix.
Add an egg, 1/4 cup of milk, a tablespoon of butter and vanilla (if you please).
Mix.
Spoon into equal portions and cook over medium heat in a lightly oiled pan.

\subsubsection*{The basic method for stewing}

This works well for several kinds of meat, including beef and pork (the author's preferred meats, rather than, say, chicken).

Take a small bowl of flour and add salt, then flour meat.
Brown the meat in the same pot you intend to cook the stew in, with a generous helping of butter in the pot.
You may use vegetable oil instead, but it will be worse.
Remove the meat once browned on two sides and set aside.

Add an onion (and celery, if desired/it is to be had) and cook until translucent and soft; don't let it brown.
Once this is done, add whatever herbs and spices you intend to use---for an herby flavor, oregano, thyme, and rosemary work well.
Almost every stew benefits from a couple of bay leaves.
Let this cook a minute, then add the wine.
The type of wine depends on the meat: pair as you usually would.

Once the wine has reduced a bit, add the meat and the broth, and simmer for about an hour.
Then add whatever vegetables you please (potatoes and carrots are always a welcome addition), and simmer for another half hour, or until the vegetables are done to your liking.

\subsubsection*{To make chili the Texas way}

\end{document}

